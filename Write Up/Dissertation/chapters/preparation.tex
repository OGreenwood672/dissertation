\label{sec:2}
Well-motivated project with success criteria well-justified.
Challenging and well-presented background covering Comp Sci topics beyond Part IB.
Good requirements analysis, justified selection of suitable tools, good engineering approach.

\section{Project Overview}
\label{sec:2.1}



\subsection{Simulation}
\label{sec:2.1.1}

let []

A world is a set of agents and worlds.

A station is at an immovable point for as long as the world exists.
A station can be a allow for ingredients to be collected and dropped off.
On initalisation of the world, we generate an array of stations,
or if we get a low

Agents have a location, inventory Analysis.



% \subsection{Communication Pipeline}
% \label{sec:2.1.2}

% No Communicationion



% Discrete Communication

\subsection{System Architecture}
\label{sec:2.1.3}

% Insert figure

% Similar to proposal
The program is decoupled into a frontend for visualisation and a backend controller.
The SimWrapper manages a set of vectorised worlds and broadcasts each world's state to the frontend via a websocket.
Each world involves agents with the ability to move in the four cardinal directions and interact with their nearest agent/station.
The agents can pick up/drop off specified ingredients to other agents or stations. Their goal is to work together to make and drop off a burger.
This task is complex as it involves sequential dependencies.
Agents must learn to navigate the grid world to locate specific stations and identify agents holding required ingredients.
The SimWrapper batches observations from each agent in the worlds and a global observation from each world and sends them to the controller.
The controller then feeds these observations into the actor-critic network, which returns critic values (for training), an action for every agent and a communication vector.
When using the continuous protocol, each agent receives all the communication vectors from all other agents in the same world.
When using the emergent protocol, each communication vector is quantised first before being broadcast to all other agents.
Once a pre-defined number of simulations have been run, the data that we recorded in the buffer is used to update the actor-critic network.


\section{Reinforcement Learning}
\label{sec:2.2}

Basic formula?

Idea behind rewards

Whats important
spare reward problem - how we mitigate
credit assignment problem - how we mitigate
3rd problem - what is it? - u get it

Normalisation of rewards


\subsection{POMDP}
\label{sec:2.2.1}

What is this lol?
P... O... Markov Decision Process

\subsection{PPO}
\label{sec:2.2.2}

TRPO -> PPO
Why used in MARL

Show formula

\section{Recurrent Neural Networks (LSTMs)}
\label{sec:2.3}

\section{Vector Quantisation Theory}
\label{sec:2.4}

Yknow what to explain

\section{Requirements Analysis}
\label{sec:2.5}

\subsection{Success Criteria}
\label{sec:2.5.1}

\subsection{MoSCoW Analysis} % funcational requirements
\label{sec:2.5.2}

\section{System Design \& Tool Selection}
\label{sec:2.6}

\subsection{Software Development Model}
\label{sec:2.6.1}

\subsection{Tool Selection \& Justification}
\label{sec:2.6.2}

Rust env - why? - vecotrised, talk about python GIL (global lock)
Why rust over cpp, 

Python - PyTorch - why not tensorflow + keras

Do we need to talk about VsCode, linters, pyproject.toml and libraries?

\subsection{Version Control System}
\label{sec:2.6.3}

git

\section{Starting Point}
\label{sec:2.7}

Sci-kit learn from Peters' Research
Genetic Algorithm
Coding Train lol

Camb Courses
MLRD
AI
Introduction to Probability

Maybe talk about
Information Theory
MLBI
DNNs

\section{Summary}
\label{sec:2.8}
