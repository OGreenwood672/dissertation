\documentclass[20pt]{article}

\usepackage{sectsty}
\usepackage{graphicx}

% Margins
\topmargin=-0.45in
\evensidemargin=0in
\oddsidemargin=0in
\textwidth=6.5in
\textheight=9.0in
\headsep=0.25in

\begin{document}


%--Paper--
\section*{\fontsize{30}{36}\selectfont Proposal}

\section*{Notes}
This work compares emergent discrete communication—using quantized message protocols—and continuous communication, both managed through Graph Attention Networks (GAT), for complex multi-agent reinforcement learning (MARL) tasks built on the QMIX framework. By integrating GATs to govern which agents communicate and how, the study systematically evaluates how discrete channel constraints (e.g., using vector quantization) versus raw continuous message sharing affect coordination, learning efficiency, and overall task performance in challenging, cooperative environments.

\section*{\fontsize{15}{5}\selectfont MARLo Polo: Emergent Communication in Multi-Agent Reinforcement Learning for Complex Tasks}

\section{Introduction}


Since Google Deepmind released AlphaGo in 2016, there has been a steady increase of interest into the field of reinforcement learning. This has led to breakthroughs in areas of finance, trading, robotics and even video games. A clear next step for reinforcement learning is multi-agent reinforcement learning where multiple agents are placed into the same environment and have to learn to interact with one another efficiently. This area of research has potential to improve surgery robots, traffic optimisations and search and rescue operations. However, there exist a range of challenges which make multi-agent reinforcement learning more difficult in practice such as: non-stationary problem, credit assignment issues, and the one I will focus on in this project, communication problems. Unlike single-agent reinforcement learning, multi-agent reinformcent learning has the ability to communicate to improve the agent's knowledge on the environment and/or teach them the optimal next step. A very common issue with communication is the "Joint Exploration Problem", this is a struggle for agents to communicate effectivly in novel environments. A recent paper[1] described a technique which is able to counter this issue was suggested.



In this project, I propose to extend AI Mother Tongue to a realistic situation for a more rigourous evaluation of the paper's proposed method. I will compare the emergent communication to a simplistic communication protocol for a quantitive comparison.


Subgoals focusing on larger goals instead of technical goals. Like We must produce a simulation to test our protocols on which can swap communication protocols.



\pagebreak
\section{Structure of the Project}
There are five main components to this project:

\begin{itemize}
\item Simulation: Building 
\item Value Decomposition Network
\item Communication Protocols
\item Training
\item Evaluation: refer to Evaluation section
\end{itemize}

\section*{Simulation}


\section*{Value Decomposition Network}

\section*{Communication Protocols}

\section*{Training}


\section{Evaluation}

\subsection*{Quantitive}


\subsection*{Qualitive}



\section{Starting Point}





\section{Success Criteria}




\section{Possible Extensions}

\subsection{HQ-VAE}

\subsection{GAT}

\section{Timetable and Milestones}

\subsection*{Weeks 1 to 2}
Proposal submitted

\subsection*{Weeks 3 to 4}

\subsection*{Weeks 5 to 6}

\subsection*{Weeks 7 to 8}

\subsection*{Weeks 9 to 11}

\subsection*{Weeks 12 to 13}

\subsection*{Weeks 13 to 15}

\subsection*{Weeks 16 to 17}

\subsection*{Weeks 18 to 19}

\subsection*{Weeks 20 to 21}

\subsection*{Weeks 22 to 23}
\subsection*{Weeks 24 to 25}

\subsection*{Weeks 26 to 27}
\subsection*{Weeks 28 to 29}

\section{Resource Declaration}

 I will be using my personal desktop computer (AMD Ryzen 7 3700X, 32GB RAM, NVIDIA RTX 5060 Ti) as my primary
 machine for software development. As a backup, I will use my personal laptop and resources provided by SRCF (student run computing facility). I will continuously backup my code and dissertation with Git version control onto GitHub.

\pagebreak
\end{document}